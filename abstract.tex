Title: Protein Alignment for Functional Profiling Whole Metagenome Shotgun Data. (allowed 10 words or 90 characters)

Abstract: Whole metagenome shotgun sequencing is a potentially powerful approach to assaying functional potential of microbial communities.  The process is currently limited by the lack of tools that can efficiently and accurately map DNA sequence reads to functionally annotated reference genomes.  Here we present a modification of the Burrows Wheeler Alignment that significantly improves the efficiency and accuracy of functional read mapping by directly mapping in protein sequence space. (3 sentences, allowed 70 words)
\textbf{Motivations:} To explore methods to rapidly and accurately identify protein coding genes from mock metagenome reads, we developed a novel protein aligner. By read mapping through degenerate nucleotide space (via NovoAlign) and protein space (via BWA), we sought to expand the potential list of candidate genes recognized in a mock metagenomic dataset.\\
\textbf{Results:} Our modified NovoAlign script was initially used to compare percent reads mapped successfully between reads from one species and a single reference genome to determine differences between standard and modified read mapping parameters across a range of percent divergence between read and reference. This degenerate nucleotide approach provided significant improvements in percent reads mapped when reference genomes were between 3 - 15 \% divergent. Next. we used the modified NovoAlign program to read map a mock metagenomic dataset to a single reference and found the following results. These include differences in percent reads mapped when comparing the original one-to-one result with the concatenated set; these include reporting the number of false positives. \\

Our second method supported aligning reads to references in protein space.  We implemented the Protein Alignment and Detection Interface (PALADIN), extending the Burrows-Wheeler Alignment (BWA) codebase to operate in protein space, coupled with other novel functionality to support the modified alignment process.  While  the degenerate nucleotide approach showed an increase in read sequence alignment within the aforementioned divergence range, the protein alignment method consistently mapped a greater number of reads than BWA for divergence greater than 1\%, with significant improvements for divergence greater than 5\%.  \\

These results suggest that alternatives to traditional read mapping packages can be readily employed to improve the breadth of gene targets discovered among novel metagenomic datasets. Furthermore, by using an appropriate representative set of reference genomes, sequence alignment within the protein space also offers a lightweight, significantly faster alternative to database search algorithms like BLAST for alignment between homologous genes.  By implementing this technique within a proven, mature, and efficient framework like BWA, these combined advantages offer an attractive solution for the read alignment of metagenomic datasets.

  