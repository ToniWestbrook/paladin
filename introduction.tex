\section{Introduction} 
The analysis of microbial community composition and function has advanced rapidly with improved sequencing technologies.  The analysis of community composition or “who is there” has advanced by focusing on a small number of phylogenetically informative gene surveys using loci such as the small subunit of ribosomal RNA (SSU rRNA). This ability to profile community structures dramatically advanced our understanding of taxonomic diversity and contributed many new insights into the potential role of the human microbiome in human health (refs).  Nevertheless, the single gene survey approach has limited taxonomic resolution, grouping together microorganisms with significantly different genomic functional capabilities. 
Functional profiling or “what can these organisms do?” based on Whole Metagenome Shotgun (WMS) data strives to assay the catalog of genes (e.g. proteins) present in a community.  However, the utility of random DNA sequence reads from metagenome is limited by the incomplete nature of annotated reference databases.  Specifically, rapid methods for mapping DNA sequence reads to annotated reference genes fail when the references within the curated databases moderately diverged from the sequences of homologous genes in metagenomics samples.  The dramatic increase in high quality sequencing output (hundreds of millions of reads with lengths routinely reaching 250 bases) has the potential to provide evidence of genes/transcripts over a large dynamic range and specifically to show much greater sensitivity in true metagenomics samples where many organisms and/or transcripts are low abundance and unlikely to assemble.  
In order to improve the mapping of DNA sequence reads to homologous protein coding DNA sequences in reference databases we propose to take advantage of the fact that protein coding DNA sequences are typically constrained by purifying selection such that divergence at silent sites in the protein coding DNA sequences outpaces divergence at nucleotide positions where a mutation will result in amino acid changes.  As a consequence read mapping that takes advantage of the constraints of protein coding function and ignoring the rapidly diverging positions has the potential to significantly increase efficiency of matching a read to a protein coding region of a reference without diminishing accuracy.  This kind of mapping can be implemented in multiple ways. First we could map DNA reads to DNA references ignoring any differences at silent sites in the coding DNA/RNA sequences.   In fact, to our there is at least one existing mapping algorithm (Novalign) that can implement this approach but has yet to be used in this context. Alternatively, we could map the putative amino acid sequences encoded by the WGS reads directly to reference protein sequences.  To our knowledge no method exists to directly map amino acid sequences to reference proteins.  Here we have modified the popular BWA mapping algorithm to allow similar performance for mapping the translations of WMS reads to reference protein coding sequences.  
To test the efficacy of our method we first derived a mock metagenomic data set of 6 standard well annotated reference genomes with diverse gene contents and base compoitions.  Read datasets were generated using ART (reference) and these were mapped to the reference coding sequences or protein sequences represented by the Uniprote protein database (Huang et al., 2012).  For the mapping of DNA sequences reads to the coding sequences for the reference proteins we used BWA (Li and Durbin, 2009) and Novoalign (Novocraft.com) (figure 1A)  In this case the maximum proportion of the reads that mapped was between X and Y.  Since the primary goal of this approach is to define the potential coding functions in WMS read data we evaluated the success based on the efficiency (percent of reads mapped) and the accuracy and sensitivity of the functional match (based on the degree of matching of the gGO terms between the source of the read and the protein to which it mapped  (see supplemental methods).  Specifically because we know the GO terms associated with the read and those of the gene where they mapped we can quantify the successful knowledge derived by the approach.
 In the second experiment we used Novoaline to map the same reads to the same coding sequences references but in this case we converted into their degenerate sequences based replacing any potentially silent nucleotide position with its degenerate based using IUPAC codes.  To our knowledge Novalign is the only algorithm that allows for read mapping to degenerate references.  The same methods were used to evaluate accuracy and sensitivity.  Describe results.
Finally we implemented a new algorithm the Protein Alignment and Detection Interface (PALADIN), that extends the highy efficient and well tested Burrows-Wheeler Alignment (BWA) codebase to operate in protein space.  In this process potential proteins encoded by the reads are mapped to the same set of proteins represented by the DNA to DNA  mapping tests.  As anticipated, the protein alignment method consistently mapped a greater number of reads than BWA, Novoalign or Novoalign degenerate for divergence greater than 1%, with significant improvements for divergence greater than 5%. ACCURACY and SENSITIVITY.  
Modification for improved read ORF prediction. 
The ability to functionally map WMS read comes at a cost of taxonomic resolution.  The proportion of reads mapped to the correct (most closely related) reference taxon was X % for the BWA and Native Novalign whereas the correct taxonomic reference was Y for the Degenerate reference mapping using Novaline and Z for the mapping in amino acid space using PAILIDIN.


  